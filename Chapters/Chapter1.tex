% Chapter 1

\chapter{Descripción del proyecto} % Main chapter title

\label{Chapter1} % For referencing the chapter elsewhere, use \ref{Chapter1} 

%----------------------------------------------------------------------------------------

% Define some commands to keep the formatting separated from the content 
\newcommand{\keyword}[1]{\textbf{#1}}
\newcommand{\tabhead}[1]{\textbf{#1}}
\newcommand{\code}[1]{\texttt{#1}}
\newcommand{\file}[1]{\texttt{\bfseries#1}}
\newcommand{\option}[1]{\texttt{\itshape#1}}

%----------------------------------------------------------------------------------------

\section{Descripción del problema}
Tomando en cuenta las soluciones HPC que provee la Universidad tecnológica de Bolívar a sus estudiantes y profesores que están encaminados a la investigación, haciendo uso de la infraestructura actual, que cuenta con 4 servidores de alto procesamiento, \textbf{Spider}, \textbf{Spider01}, \textbf{Spider02}, \textbf{Spider03}, pero que en algunos casos se quedan sin poder llevar a cabo solicitudes para nuevos trabajos.

Para estas nuevas solicitudes existe la necesidad de hacer uso de recursos adicionales con los que cuenta la universidad. A nivel de infraestructura se cuenta con laboratorios donde es posible hacer uso de máquinas de propósito general en estados de ocio, las cuales se pueden agregar a un cluster para poder aprovechar dichos recursos. 

HTCondor, nace como una herramienta de control y despliegue de cluster, lo que permite un mayor manejo sobre las máquinas que hacen parte de este, y sobre las diferentes posibles tareas que se puedan ejecutar en el mismo. La universidad Tecnológica de Bolívar, al ser líder en la implementación de nuevas tecnologías, busca permitir a profesores como a los estudiantes un recurso (cluster) en el cual puedan llevar a cabo tareas de computación de alto desempeño, mostrando así, la capacidad, la infraestructura, y el liderazgo que posee en el campo de las nuevas tecnologías.\\

%----------------------------------------------------------------------------------------
%	SECTION 1
%----------------------------------------------------------------------------------------

\section{Objetivos}
\subsection{Objetivo General}
\begin{itemize}
    \item Desplegar un cluster oportunista anexo al laboratorio de computación de alto desempeño de la Universidad Tecnológica de Bolívar, HPClab para el envío de trabajos de cómputo a diferentes entornos de ejecución, haciendo uso de GPU/CPU disponibles en máquinas en estado de espera en los laboratorios del campus.
\end{itemize}

\subsection{Objetivos específicos}
%-----------------------------------
\begin{itemize}
    \item Crear una guía para la implementación de un cluster heterogéneo mediante el uso de HTCondor en sistemas operativos Linux, Windows y OSX. \autocite{HTCondor}
    \item Documentar los casos donde se puede hacer uso del cluster oportunista, y colocar esta información en la Wiki del laboratorio de alto desempeño para que este disponible al alcance de todos.
\end{itemize}








